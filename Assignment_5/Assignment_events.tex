% ----------------------------------------------------------------
% Article Class (This is a LaTeX2e document)  ********************
% ----------------------------------------------------------------
\documentclass[12pt]{article}
\usepackage[english]{babel}
\usepackage{amsmath,amsthm}
\usepackage{amsfonts}
\usepackage{commath}
\usepackage[pdftex]{graphicx}
\usepackage[a4paper,margin=2.0cm,footskip=.5cm]{geometry}
\usepackage{float}
\usepackage{fancyhdr}
\usepackage[colorlinks=true,
	linkcolor=MidnightBlue,
	urlcolor=BlueViolet,
	citecolor=MyGreen]{hyperref}
\usepackage{listings}
\usepackage[detect-weight,detect-mode]{siunitx}
\usepackage[dvipsnames]{xcolor}

\lstset{
	%        backgroundcolor=\color{lightgray},
	basicstyle=\footnotesize\ttfamily, % Standardschrift
	breaklines=true,            % Zeilen werden Umgebrochen
	extendedchars=true,         %
	frame=b,
	frame=none,
	framexleftmargin=17pt,
	framexrightmargin=5pt,
	framexbottommargin=4pt,
	framextopmargin=0pt,
	keywordstyle=\color{blue},
	%        keywordstyle=[1]\textbf,    % Stil der Keywords
	%        keywordstyle=[2]\textbf,    %
	%        keywordstyle=[3]\textbf,    %
	%        keywordstyle=[4]\textbf,   \sqrt{\sqrt{}} %
	%		 numbers=left,               % Ort der Zeilennummern
	language=python,
	numberstyle=\tiny,          % Stil der Zeilennummern
	numbersep=5pt,              % Abstand der Nummern zum Text
    upquote=true,
	stringstyle=\color{BrickRed}\ttfamily, % Farbe der String
    commentstyle=\color{MyGreen},
    % commentstyle=\color{MyGreen},
	showspaces=false,           % Leerzeichen anzeigen ?
	showstringspaces=false      % Leerzeichen in Strings anzeigen ?
	showtabs=false,             % Tabs anzeigen ?
	%		 stepnumber=2,               % Abstand zwischen den Zeilennummern
	tabsize=2,                  % Groesse von Tabs
	%		texcl=true,
	xleftmargin=17pt,
}

\lstloadlanguages{python}
\pagestyle{fancy}
\fancyhf{}
\rhead{Assignment 5: Event Finding}
\fancyfoot[C]{\thepage}
\fancyhead[L]{\setlength{\unitlength}{1mm}
    \begin{picture}(0,0)
        \put(0, -1){\includegraphics[width=1cm]{Resources/img/bsa.png} BSA, WS 22/23}
    \end{picture}
} 

% ----------------------------------------------------------------
\begin{document}

\section*{Assignment 5: Event Finding}

\section*{First steps}

Understand the code in section 6.1 of the book \emph{Hands-on Signal Analysis
with Python}. Make sure that you understand:

\begin{itemize}
    \item How to create a logical vector by thresholding
    \item How to use logical indexing
    \item How to combine logical vectors with bitwise logical operators
    \item Why it can be useful to take the derivative before finding features
    \item How to use \lstinline{plt.subplots}
\end{itemize}


\section{Simulate experimental data}%

Generate a vector which looks like the data in Figure 6.3
in the book, and save it to the file \texttt{HorPos.txt} (with no header). If you
do that correctly, you should be able to run the file \\
\lstinline{event_detection.py}, generating a figure similar to Figure
6.3 in the book.

For \lstinline{event_detection.py} to run smoothly, your data vector should have
the following properties:

\begin{itemize}
    \item 500 points of random noise about 0 (a vector of 500 random data, normally
        distributed about 0 with a standard deviation of 1, can be generated
        with \lstinline{np.random.randn(500)}),
    \item then 500 points of values with random noise with an average value of
        -15,
    \item then the same thing for average values up to +15, in steps of 5.
    \item And at the end, again 500 points of random noise about 0.
\end{itemize}


\section{EMG Activity}

\subsection{Background}

EMG-data are some of the most common signals in movement analysis. But
sometimes the data analysis is not that simple. For example, data can be
superposed by spurious drifts. And short drops in EMG activity can obscure
extended periods of muscle contractions.

The data in \lstinline{Shimmer3_EMG_Calibrated.csv} have been taken from \\
\url{https://www.shimmersensing.com/support/sample-data/}, where also the data
description is given in detail. The first column of EMG-data describes the
muscle activity in the forearm, the second one the activity in the biceps.
Sample rate is 512 Hz, and the data are in \emph{mV}. 

\subsection{Problem specification}

 Write a function that does the following:

 \begin{itemize}
     \item Import the EMG data from the data file
         \lstinline{Shimmer3_EMG_Calibrated.csv}. Thereby the command
         \texttt{pd.read\_csv} can be used with the parameter
         \texttt{delim\_whitespace=True}, to ensure that any mixture of
         white-spaces is taken as a single separator.

         Select the column corresponding to the EMG of the forearm.

    \item Remove the offset of the EMG-recording with a Butterworth highpass
        filter.

    \item Rectify the data, and smooth them with a Savitzky-Golay filter to
        produce a rough envelope of the signal.

    \item Interactively select a threshold, and use this to find the start- and
        end-points of muscle activity, using the \texttt{numpy} command
        \texttt{where}. Watch out that activities that last less
        than 0.5 sec are probably measurement artefacts!
 
    \item Eliminate further artefacts, by cutting away transients at the beginning of
        the file.
    \item Calculate and display the mean contraction time.

\end{itemize}


 \section{Gait Events}

Heel Strike and Toe Off are two important events within the human gait, and give
the examined an insight in to the timing during walking. Their occurrence can be
measured using a force plate.

\begin{itemize}
    \item Load \lstinline{GroundReactionForce.mat}, which can be found in the
        directory \lstinline{Data}.
    \item Plot the z-axis of the data
    \item Choose a threshold to detect when the foot is on the force plate
    \item Set a green marker when an heel strike occurs
    \item Set a red marker when an toe off occurs
\end{itemize}


\section{R-peaks}

R-peaks are the most prominent events in an ECG signal, therefore they can be
used to calculate the heart rate.

\begin{itemize}
    \item Load \lstinline{ecg-hfn.mat} (fs=1000Hz) , which can be found in the
        directory \lstinline{Data}.

    \item Plot the z-axis of the data

    \item Find R-peaks automatically and highlight them in the plot with a
        marker.\\ (Hint: find zero-crossings of the $1^{st}$ derivative, by
        checking where the product of two subsequent velocities is negative, and
        the ECG is simultaneously higher than a selected threshold.)

    \item Calculate the mean and standard deviation of the heart bpm
        (beats-per-minute). Be careful to note that the heart rate in
        \texttt{[Hz]} is giving the beats per second!!
\end{itemize}

\end{document}
